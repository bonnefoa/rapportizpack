Le projet visait à supprimer la dépendance de IzPack à NanoXML.
Cet objectif à été réalisé, la librairie NanoXML a été entièrement supprimé du projet.
Ces changements ont été placé sur la branche principale et seront disponible pour la 4.3 de IzPack, qui sortira le 24 avril.


Les tests assurent la fiabilité de la solution mais des bugs peuvent subsister.
C'est pour ça que nous continuons de suivre nos réalisations hors du cadre du projet en devenant contributeur.
De plus, nous participons désormais activement à l'évolution du projet.

Il reste encore de nombreuse possibilités d'amélioration pour IzPack.
La suppression de NanoXML en faisait partie.
Nous travaillons actuellement sur d'autres améliorations pour améliorer la qualité de IzPack.
En effet, il est préférable pour les nouveaux contributeurs de faire des petites modifications afin de b
