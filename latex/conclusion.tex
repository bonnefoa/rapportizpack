Le projet visait à supprimer la dépendance de IzPack à NanoXML.
Pour cela, nous avons étudié l'existant et l'utilisation faite dans IzPack de NanoXML.
Nous avons ensuite cherché une solution pour supprimer cette dépendance.
Le patron de conception adaptateur était particulièrement adapté.

Nous avons alors procédé aux développement de cette solution en utilisant Git et IntelliJ pour nous assister.
La solution a ensuite été testé pour s'assurer du bon fonctionnement de celle-ci.
Après quoi, elle a été intégrée dans IzPack.

Ces changements ont été placé sur la branche principale et seront disponible pour la 4.3 de IzPack, qui sortira le 24 avril. Les tests assurent la fiabilité de la solution mais des bugs peuvent subsister.
C'est pour ça que nous continuons de suivre nos réalisations hors du cadre du projet en devenant contributeur.
De plus, nous participons désormais activement à l'évolution du projet.

Il reste encore de nombreuse possibilités d'amélioration pour IzPack.
La suppression de NanoXML en faisait partie.
Nous travaillons actuellement sur d'autres améliorations pour améliorer la qualité de IzPack.

