% \section{État au commencement du projet}
Le projet etant open-source et actif, il evolue sans cesse. La partie sur laquelle nous avons travaille est relativement stable mais son utilisation peut varier dans le code. Nous allons voir la portee de l'utilisation de nanoXml, comment il est utilise et enfin la solution apportee.

% -> Utilisation intensive de XML
% -> Philosophie de nanoXml et de DOM differente
% -> -> DOM : Tout est noeud
% -> -> NanoXml : Base sur les tags (elements)
\section{Nanoxml}
Comme nous avons pu le voir, IzPack utilise intessement des fichiers XML pour gerer la generation d'installations. En effet, les fichiers XML presentent l'avantage d'etre lisible, ecrivable a la main et traitables. Pour traiter ces fichiers, une librairie a ete utilisee, il s'agit de NanoXml.
\subsection{Le projet NanoXml}
NanoXml est une librairie open-source de gestion de Xml java. Elle presente la particularite d'etre legere (~108Ko). Il est possible, grace a cette librairie, d'ecrire ou de parser des fichiers Xml. Une fois le fichier parse, on recupere en memoire la structure de l'arbre Xml qui peut alors etre exploite. Le choix de cette librairie a ete faite au commencement de Izpack a cause de sa taille. En effet, il ete important de ne pas surcharge l'application a installe par un installeur imposant. 

Cependant, ce projet n'est plus de mise a jour depuis 7 ans. Entretemps, java integre depuis la version 1.4 un parser Xml et des bugs ont ete decouvert. Il devenait interessant de remplacer NanoXml par Jaxp(Java API for XML Processing), la librairie fournie par la machine virtuelle.
\subsection{Les differentes composantes de NanoXml}
On trouve differents elements dans NanoXml.
\subsubsection{StdXMLParser}
Le parser permet de creer un arbre logique representant le fichier xml qui lui a ete passe. Il est systematiquement utilise au moment de la lecture du fichier afin de recuperer l'arbre.
\subsubsection{XMLElement}
Le XMLElement represente un element logique du xml en memoire. Par exemple :
\begin{verbatim}
<root>
	<enfant>
		content
	</enfant>
</root>
Nano -> root possède 1 enfant
DOM -> root possède 3 enfants (\n; <noeud1>; \n)
\end{verbatim}
En parsant un fichier contenant ce Xml, on a un XMLElement nomme root qui possede 1 enfant. Son enfant est un XMLElement ne dont le contenue est content.

L'arborescence du XML est donc realise par un ensemble de XMLElement. XMLElement permet de recuperer la liste des enfants, le contenu, recherche un enfant par son nom, retirer un enfant, ajouter un enfant ou modifier le contenu. Il s'agit de loin de la classe la plus utilisee de la librairie NanoXml dans Izpack. En effet, on compte pas moins de 499 utilisations au commencement de ce projet.
\subsubsection{XMLWriter}
Le XMLWriter permet d'ecrire des fichiers XML. Il reproduit, a partir d'un XML element, le fichier XML correspondant. Il n'est utilise que pour generer un xml temporaire ou pour ecrire le xml d'installation automatique. 
\subsection{Utilisation typique dans Izpack}
\subsubsection{Parse de fichier}
Le parse du fichier est execute pour recuperer le XML.
\begin{verbatim}
        StdXMLParser parser = new StdXMLParser();
        parser.setBuilder(XMLBuilderFactory.createXMLBuilder());
        parser.setReader(new StdXMLReader(in));
        parser.setValidator(new NonValidator());

        // We get the data
        XMLElement data = (XMLElement) parser.parse();
\end{verbatim}
\subsubsection{Ecriture d'un xml}
\begin{verbatim}
       XMLWriter writer = new XMLWriter(out);
        ...
        writer.write(xmlElemetn);
\end{verbatim}
\subsubsection{Utilisation de XMLElement}
\begin{verbatim}
String id = el.getAttribute("id");
String packImgId = el.getAttribute("packImgId");
boolean loose = "true".equalsIgnoreCase(el.getAttribute("loose", "false"));
String description = requireChildNamed(el, "description").getContent();
boolean required = requireYesNoAttribute(el, "required");
String group = el.getAttribute("group");
String installGroups = el.getAttribute("installGroups");
String excludeGroup = el.getAttribute("excludeGroup");
boolean uninstall = "yes".equalsIgnoreCase(el.getAttribute("uninstall", "yes"));
String parent = el.getAttribute("parent");
String conditionid = el.getAttribute("condition");
\end{verbatim}
\section{Jaxp}
Jaxp, le processor Xml de Sun, respecte les normes W3C. Il y a deux manieres d'exploiter les XML avec Jaxp, soit en utilisant le parseur Sax, soit en utilisant le parseur DOM.
\subsection{SAX}
SAX (Simple API for Xml) est une API pour parser les XML. Cette API possede la particularite de proceder par evenement. C'est a dire que le parser va lire le fichier ligne par ligne et envoye un evenement pour chaque element particulier qu'il rencontre. 

On a des evenements de commentaires, de text ou d'elements. L'avantages de cette API est l'utilisation memoire faible, comme le xml n'est pas mis en memoire mais lu et traite au fur et a mesure. Il permet de traiter sans probleme les fichiers important. En contrepartie, le traitement et la facilite de manipulation est plus difficile.
\subsection{DOM}
L'interface DOM(Document Object Model) de jaxp propose de mettre en memoire le xml et offre une interaction basee sur des nodes. En DOM, tout est nodes. Les commentaires, le texte ou les elements sont representes par des noeuds. Des sous-classes specialisent ces noeuds. comme Comment, Element ou Text.

La manipulation des elements du xml se fait grace aux nodes. Il est possible de recuperer le voisin d'une node, d'enlever ou d'ajouter un enfant et de recuperer ou de modifier le contenue.
\section{Elaboration de la solution}
Face a ce probleme, notre travail a ete de chercher une solution pour permettre de remplacer de NanoXml et de l'appliquer au projet. Mais ils y avaient plusieurs problemes a cela.
\subsection{Les differences entre Jaxp et NanoXml}
Comme nous avons vu precedement, les 2 api ne proposent pas la meme approche pour le traitement des Xml. Nous avons choisies d'utiliser l'interface DOM vu que les avantages de SAX ne sont pas necessaires et qu'il etait peut concevable de traiter les XML avce SAX dans IzPack.

Nous allons donc comparer l'interface DOM et NanoXml pour determiner la meilleure approche a prendre.
\subsubsection{Parse du XML}
\subsubsection{Utilisation du XML}

\begin{verbatim}
<!--comment>commentaire<comment-->
<root>
	<noeud1>text</noeud1>
</root>

Nano -> root possede 1 enfant
DOM -> root possede 3 enfants (\n; <noeud1>; \n)
\end{verbatim}
\subsection{Remplacer les appels}
\subsection{Utilisation d'un design pattern}
Extraction d'interface du xmlElement.
-> IXmlElements
-> IXmlParser
-> IXmlWriter

XMLElements de NanoXml ~ Node Element

Dom possède plusieurs types de noeuds : Comment, Text, element

Xinclude : géré de maniere native par Jaxp
Xfragment Non present dans la specifiaction W3C:
-> Permet include de noeuds sans racines
Passage par une feuille XSL
-> include puis suppression des Xfragment
\section{Application et tests}
\subsection{Test unitaires de comparaison}
Comparaison entre le comportement de nanoXml et de l'adaptateur
\subsection{Integration a IzPack}
\subsection{Test Globale et Profiling}

\section{Resultats}
\subsection{Diminution de la taille des installeurs}
NanoXml : 116Ko
Adaptateur : 44Ko
\subsection{Support de fonctionnalitees XML}
DTD, xinclude

\section{Au delà du sujet}
Une fois l'objectif du projet atteint, nous avons décidé d'accepter l'offre de Julien Ponge en devenant contributeurs officiels du projet. Au même moment, nos apports étaient intégrés au dépot officiels et se retrouveront dans la prochaine version 4.3.

\subsection{Suivi de nos apports}
Etant désormais développeur sur le projet IzPack nous avons pu suivre et corriger les problèmes rencontrés par la communauté de IzPack.

\subsection{Correction de bugs}

\subsection{Ajout/Amélioration}