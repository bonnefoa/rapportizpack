\section{État au commencement du projet}
-> Utilisation intensive de XML
-> Philosophie de nanoXml et de DOM différente
-> -> DOM : Tout est noeud
-> -> NanoXml : Base sur les tags (éléments)
\begin{verbatim}
<!--comment>commentaire<comment-->
<root>
	<noeud1>text</noeud1>
</root>

Nano -> root possède 1 enfant
DOM -> root possède 3 enfants (\n; <noeud1>; \n)
\end{verbatim}
\subsection{Présence de Nanoxml}
Comme nous avons pu le voir, IzPack utilise intensement des fichiers XML pour gérer les générations d'installations. En effet, les fichiers XML présentent l'avantage d'être lisible, ecrivables et facilement traitables. Pour traiter ces fichiers, une librairies a été utilisée, il s'agit de NanoXml.

NanoXml est une librairie de gestion de Xml java. Elle présente la particularité d'être légère (~150Ko). Il est possible, grâce a cette librairie, d'écrire ou de parser des fichiers Xml. Une fois le fichier parse, on récupère en mémoire la structure de l'arbre Xml qui peut alors être exploite.

\subsection{Utilisation de Nanoxml}
\begin{verbatim}
        StdXMLParser parser = new StdXMLParser();
        parser.setBuilder(XMLBuilderFactory.createXMLBuilder());
        parser.setReader(new StdXMLReader(in));
        parser.setValidator(new NonValidator());

        // We get the data
        XMLElement data = (XMLElement) parser.parse();
\end{verbatim}

\section{Élaboration de la solution}
\subsection{Remplacer les appels}
\subsection{Utilisation d'un design pattern}
Extraction d'interface du xmlElement.
-> IXmlElements
-> IXmlParser
-> IXmlWriter

XMLElements de NanoXml ~ Node Element

Dom possède plusieurs types de noeuds : Comment, Text, element

Xinclude : géré de maniere native par Javax
Xfragment Non present dans la specifiaction W3C:
-> Permet include de noeuds sans racines
Passage par une feuille XSL
-> include puis suppression des Xfragment
\section{Application et tests}
\subsection{Test unitaires de comparaison}
Comparaison entre le comportement de nanoXml et de l'adaptateur
\subsection{Integration a IzPack}
\subsection{Test Globale et Profiling}

\section{Resultats}
\subsection{Diminution de la taille des installeurs}
NanoXml : 116Ko
Adaptateur : 44Ko
\subsection{Support de fonctionnalitees XML}
DTD, xinclude

\section{Au delà du sujet}
Une fois l'objectif du projet atteint, nous avons décidé d'accepter l'offre de Julien Ponge en devenant contributeurs officiels du projet. Au même moment, nos apports étaient intégrés au dépot officiels et se retrouveront dans la prochaine version 4.3.

\subsection{Suivi de nos apports}
Etant désormais développeur sur le projet IzPack nous avons pu suivre et corriger les problèmes rencontrés par la communauté de IzPack.

\subsection{Correction de bugs}

\subsection{Ajout/Amélioration}