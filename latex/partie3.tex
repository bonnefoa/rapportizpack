\subsection{Etat au commencement du projet}
-> Utilisation intensive de XML
-> Philosophier de nanoXXml et de DOM differente
-> -> DOM : Tout est noeud
-> -> NanoXml : Base sur les tags (elements)

<!--comment>commentaire<comment-->
<root>
	<noeud1>text</noeud1>
</root>

Nano -> root possede 1 enfant
DOM -> root possede 3 enfants (\n; <noeud1>; \n)
\subsubsection{Presence de nanoxml}
\subsubsection{Utilisation de nanoxml}


\subsection{Elaboration de la solution}
\subsubsection{Remplacer les appels}
\subsubsection{Utilisation d'un design pattern}
Extraction d'interface du xmlElement.
-> IXmlElements
-> IXmlParser
-> IXmlWriter

XMLElements de NanoXml ~ Node Element

Dom possede pls types de noeuds : Comment, Text, element

Xinclude : gerer de maniere native par Javax
XfragmentL Non present dans la specifiaction W3C:
-> Permet include de noeuds sans racines
Passage par une feuille XSL
-> include puis suppression des Xfragment
\subsection{Application et tests}
\subsubsection{Test unitaires de comparaison}
Comparaison entre le comportement de nanoXml et de l'adaptateur
\subsubsection{Integration a IzPack}
\subsubsection{Test Globale et Profiling}

