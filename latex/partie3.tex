\section{Etat au commencement du projet}
-> Utilisation intensive de XML
-> Philosophier de nanoXXml et de DOM differente
-> -> DOM : Tout est noeud
-> -> NanoXml : Base sur les tags (elements)
\begin{verbatim}
<!--comment>commentaire<comment-->
<root>
	<noeud1>text</noeud1>
</root>

Nano -> root possede 1 enfant
DOM -> root possede 3 enfants (\n; <noeud1>; \n)
\end{verbatim}
\subsection{Presence de nanoxml}
\subsection{Utilisation de nanoxml}
\begin{verbatim}
        StdXMLParser parser = new StdXMLParser();
        parser.setBuilder(XMLBuilderFactory.createXMLBuilder());
        parser.setReader(new StdXMLReader(in));
        parser.setValidator(new NonValidator());

        // We get the data
        XMLElement data = (XMLElement) parser.parse();
\end{verbatim}

\section{Elaboration de la solution}
\subsection{Remplacer les appels}
\subsection{Utilisation d'un design pattern}
Extraction d'interface du xmlElement.
-> IXmlElements
-> IXmlParser
-> IXmlWriter

XMLElements de NanoXml ~ Node Element

Dom possede pls types de noeuds : Comment, Text, element

Xinclude : gerer de maniere native par Javax
XfragmentL Non present dans la specifiaction W3C:
-> Permet include de noeuds sans racines
Passage par une feuille XSL
-> include puis suppression des Xfragment
\section{Application et tests}
\subsection{Test unitaires de comparaison}
Comparaison entre le comportement de nanoXml et de l'adaptateur
\subsection{Integration a IzPack}
\subsection{Test Globale et Profiling}

\section{Resultats}
\subsection{Diminution de la taille des installeurs}
NanoXml : 116Ko
Adaptateur : 44Ko
\subsection{Support de fonctionnalitees XML}
DTD, xinclude
