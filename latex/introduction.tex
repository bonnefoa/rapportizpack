Dans le cadre de notre troisième année d'étude à l'Institut Supérieur d'Informatique, de Modélisation et de leurs Applications, nous avons effectué un projet d'une durée de 240 heures entre novembre 2008 et mars 2009 sous la tutelle de Julien Ponge.
Le sujet porte sur IzPack, un projet open-source membre de la fondation Codehaus.

Le développement de logiciel prend de l'ampleur et nécessite de plus en plus de temps.
Externaliser certaines logiques et certaines procédures dans des outils externes qui permettent de diminuer ce temps de développement devient alors intéressant.
Izpack rentre dans cette catégorie d'outils : en effet, il se charge de générer la partie installation de l'application.

Cependant, IzPack est un logiciel à part entière et nécessite un investissement.
Il rencontre comme tous les logiciels, des problèmes incontournables : bugs, librairies obsolètes, problème de design, etc.
De même, des nouvelles pratiques de programmation apparaissent alors que les anciennes montrent leurs limites.
Pour résoudre ces problèmes, une pratique courante est de faire du refactoring sur l'application.
Le refactoring consiste à retoucher le code, étape par étape, afin d'améliorer le design, la qualité et la maintenabilité du code sans ajouter de fonctionnalités ni modifier la logique existante.

Le refactoring peut-être considéré comme une perte de temps d'un point de vue extérieur étant donné qu'aucun travail visible par l'utilisateur n'est réalisé.
Cependant, cette pratique est indispensable pour assurer la pérennité de l'application.
En effet, l'ajout de fonctionnalités se fait souvent au détriment de la lisibilité.
En revanche il est possible de trouver un meilleur design après un temps de réflexion.
Le refactoring permet de corriger la première implémentation réalisée et de produire du code plus mature et plus stable.

Toutefois, cette pratique est généralement peu appliquée en raison de l'absence d'intérêt à court terme.
IzPack présente aujourd'hui des améliorations possibles réalisables grâce au refactoring, notamment dans la gestion des fichiers XML.
En effet, il s'appuie sur NanoXML pour gérer ces fichiers, mais la vetusté de cette librairie entraîne un certain nombre de problèmes.
L'utilisation des librairies natives de Java régleraient ces problèmes.

Dans le cadre de notre projet de troisième année à l'ISIMA, nous avons été chargé du refactoring de cette partie d'IzPack.
Nous verrons d'abord le fonctionnement d'IzPack, puis le cadre de travail du projet et enfin le travail réalisé et ses résultats.