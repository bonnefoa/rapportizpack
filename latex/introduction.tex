    Dans le cadre de notre deuxieme annee d'etude a l'Institut Superieur d'Informatique, de Modelisation et de leurs
Applications, nous avons effectues un projet d'une duree de 240 heures entre novembre 2008 et mars 2009 sous la
tutelle de Julien Ponge.

Ameliorer un logiciel ne se resume pas a ajouter des fonctionnalites ou corriger de bugs. Faire du refactoring permet de rendre le code plus maintenable, lisible : le vieil adage ``ca fonctionne, alors on n'y touche pas'' a mene plus d'un projet dans l'impasse du code inmaintenable. Les resultats ne sont pas visibles, car d'un point de vue exterieur rien n'a change. Pourtant pour un developpeur cela signifie un developpement present et futur plus rapide et plus aise. La supression/remplacement d'une librairie obsolete entre dans ce cas.

IzPack, un generateur d'installeur open-source possede justement une dependance de ce type avec NanoXml. Dans le cadre de notre projet de troisieme et derniere annee a l'ISIMA, nous avons ete charge du refactoring de cette partie.
% inspiration en baisse, ecriture suspendue