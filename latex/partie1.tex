\section{Decription}
IzPack est un generateur d'installeur d'application open-source cree en 2001 par Julien Ponge.
\subsection{Generateur d'installeur}
A partir d'une application et par l'ecriture d'un descripteur d'installation en XML, il est capable de generer un fichier d'installation cross-plateforme. Tout type de programme peut etre package, que ce soit une application C++, Java... Izpack permet la creation rapide d'installeur pour ces programmes.
\subsection{Open-source}
Izpack est sous license Apache2. Une importante communaute s'est regroupee autour de ce projet qui a permit son evolution jusqu'a maintenant. 

\subsection{Fonctionnalites}
Izpack possede plusieurs fonctionnalites, nous allons en voir les principales:
\subsubsection{Cross-plateforme}
Izpack cree un installeur Java. Il suffit donc que la machine ait un machine virtuelle pour pouvoir lancer l'installation, independament de la plateforme et du systeme d'exploitation.
\subsubsection{Customization}
Izpack possede un ensemble de panels qui vont constituer l'installateur graphique. L'aspect globale et l'aspect des panels est personnalisable par l'utilisateur via le descripteur XML.
\subsubsection{Internationnalisation}
Izpack supporte la creation d'installeur multilangues. 
\subsubsection{Installation automatique}
A la fin de l'installation, il est possible de generer un script d'installation automatique. Ce script permet de reproduire l'installation realisee sur d'autres machines.
\subsection{Popularitee}
Izpack est utilise dans de grand projets comme Jboss, Xwiki, Glassfish... A l'heure actuelle, les telechargements mensuels s'elevent a 15.000.
% Image des telechargement?
\section{Architecture}

\subsection{Architecture globale}
\subsection{Installeur}
\subsubsection{Exemples de panels}
\subsection{Compilateur}

\section{Exemples d'installation}
\subsection{Description du xml}
\subsection{Generation du jar}
\subsection{Installation}
\section{Problemes actuels}
\subsection{Nanoxml}