\section{Decription}
IzPack est un generateur d'installeur d'application open-source cree en 2001 par Julien Ponge.
\subsection{Generateur d'installeur}
A partir d'une application cree, Izpack est capable de generer un installateur. Cet installateur pourra etre utilise pour deployer l'application sur n'importe quelle machine. L'interet d'Izpack reside dans le fait que cree une installation est souvent laborieux et n'apporte que peu de plus-value au programme. Izpack propose une solution pour creer cet installeur de maniere simple et universelle. 

Izpack 
Tout type de programme peut etre package, que ce soit une application C++, Java...
\subsection{Open-source}
Izpack est sous license Apache2. Cette license permet l'acces au code source et l'utilisation du logicielle. Il est tout a fait possible de l'utiliser pour une application commerciale voire, de modifier les sources pour correspondre a ses besoins. Une importante communaute s'est regroupee autour de ce projet ce qui a permit son evolution jusqu'a maintenant.

\subsection{Fonctionnalites}
Izpack possede plusieurs fonctionnalites, nous allons en voir les principales:
\subsubsection{Cross-plateforme}
Izpack cree un installeur Java. Il suffit donc que la machine ait un machine virtuelle pour pouvoir lancer l'installation, independament de la plateforme et du systeme d'exploitation.
\subsubsection{Customization}
Izpack possede un ensemble de panels qui vont constituer l'installateur graphique. L'aspect globale et l'aspect des panels est personnalisable par l'utilisateur via le descripteur XML.
\subsubsection{Internationnalisation}
Izpack supporte la creation d'installeur multilangues. 
\subsubsection{Installation automatique}
A la fin de l'installation, il est possible de generer un script d'installation automatique. Ce script permet de reproduire l'installation realisee sur d'autres machines.
\subsection{Popularitee}
Izpack est utilise dans de grand projets comme Jboss, Xwiki, Glassfish... A l'heure actuelle, les telechargements mensuels s'elevent a 15.000.
% Image des telechargement?
\section{Architecture}
\subsection{Architecture globale}
Globalement, Izpack possede 2 composants, le compilateur qui va gerer la creation de l'installeur et l'installeur.
\subsection{Compilateur}
Le compilateur package l'ensemble des fichiers necessaires dans un seul jar. Selon la description de l'installation, il va incorporer les panels necessaires au lancement de celle-ci. 
\subsection{Installeur}
La partie installation concerne toute la logique et la presentation du processus d'installation. 
\subsubsection{Exemples de panels}

\section{Exemples d'installation}
\subsection{Description du xml}
\subsection{Generation du jar}
\subsection{Installation}
\section{Problemes actuels}
\subsection{Nanoxml}