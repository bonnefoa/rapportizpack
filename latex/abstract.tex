\textbf{IzPack} est un \textbf{logiciel libre} Java dédié à la création d'\textbf{installateur} personnalisé pour applications.
Il permet de décharger le développement en automatisant ce processus.
Cependant, IzPack est sujet aux mêmes problèmes que tous les logiciels et nécessitent des phases de \textbf{refactoring}.
Ces phases permettent d'améliorer la qualité du code et de le rendre plus maintenable.

Le changement d'une librairie fait partie des actions possibles pour le refactoring.
Plusieurs raisons peuvent motiver ce changement: une librairie utilisée peut ne plus être maintenue ou une autre plus robuste existe.
C'est le cas de la librairie de gestion de \textbf{XML} de IzPack, NanoXML, dont le développement à été arrété.
De plus, Java propose dans ses nouvelles versions une librairie native remplissant le même rôle, rendant NanoXML superflue.

Cette opération de refactoring représente l'objectif de notre projet.
Dans un premier temps, il nous a fallut étudier le fonctionnement d'IzPack et l'utilisation qui a été faite de NanoXML.
Après quoi, aidé du gestionnaire de version \textbf{Git} et de l'environnement de développement \textbf{IntelliJ}, nous avons élaboré une solution basée sur le patron de conception adaptateur.
Cette solution a été implémentée dans IzPack et sera présente pour la prochaine version de IzPack.
~\\
~\\
Mots-clés : IzPack, installateur, logiciel libre, XML, Git, IntelliJ, refactoring
\vfill
\begin{center}\large{\textbf{Abstract}}\end{center}

\textbf{IzPack} is an \textbf{open source software} dedicated to build custom \textbf{installer} for applications.
Thus, it can reduce the developpment of applications by automating this process.
However, IzPack has the same issues of all softwares and need \textbf{refactoring} phases.
Those phases enhance the code quality and its maintainability.

The replacement of a library is a part of refactoring actions.
Several reasons can spur this change : a used library which is no longer developped or a more efficient one exists.
This is the case of the \textbf{XML} library used by IzPack, NanoXML, which is no longer developped.
Moreover, Java provides a XML library in its recent version, which make NanoXML superfluous.

Our project's objective is this refactoring operation.
First we have studied how IzPack works and how it uses NanoXML.
Then, helped by the source control management \textbf{Git} and by the integrated developpment environnement \textbf{IntelliJ}, we developped a solution based on the design pattern adapter.
This solution has been merged with IzPack and will be in the next version of IzPack.

Keywords : IzPack, installeur, open source software, XML, Git, IntelliJ, Refactoring