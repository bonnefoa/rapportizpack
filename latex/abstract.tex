IzPack est un logiciel open source écrit en java générant des installeurs personnalisés à partir de n'importe quel logiciel.
Il utilise massivement les fichiers XML et se base sur la librairie NanoXML, librairie qui n'est plus mise à jour.
Depuis la naissance d'IzPack et le choix de NanoXML, la gestion des fichiers XML a été ajouté à Java, rendant NanoXML superflue.

Durant notre projet, il a d'abord fallut découvrir le fonctionnement de IzPack puis chercher une solution pour utiliser les librairies java standards à la place de NanoXML.
Pour nous aider dans cette tâche, nous avons utilisé plusieurs outils, git comme gestionnaire de version, IntelliJ comme IDE, etc.
Une solution a été trouvée et mise en \oe{}uvre grâce au patron de conception adaptateur.
Cette solution, largement testée, sera intégrée dans la prochaine version de IzPack.
~\\
~\\
Mots-clés : Java, logiciel libre, xml, connard, IntelliJ
\vfill
\begin{center}\large{\textbf{Abstract}}\end{center}

IzPack is an open source software written in java which generate custom installers based on any software.
It massively uses XML files and is based on the NanoXML library, a no more updated one.
Since the IzPack's birth and the choice of NanoXML, the handling of xml files has been added to Java, so NanoXML is now superfluous.

During our project, we had to understand how work IzPack and then to find a solution to use java standard librairies instead of NanoXML.
to help us in this task, we have used several tools : ``git'' as version control system, IntelliJ as IDE, etc.
A solution has been found and implemented thanks to the adapter design pattern.
this solution, heavily tested will be integrated in the next version of IzPack.
~\\
~\\
Keywords : Java, open source software, xml, git, IntelliJ