\begin{description}
	\item[packager] Packager est l'action de regrouper en un seul point plusieurs ressources.
	\item[parser] Parser est l'action de lire un fichier contenant des informations structurées pour les charger en mémoire.
	\item[refactoring] C'est une modification du code avec pour but de le rendre plus lisible, maintenable, testable. Il n'ajoute pas de fonctionnalités.
	\item[design pattern] Patron de conception.
	\item[XML] eXtensible Markup Language. C'est un langage descriptif, qui permet de decrire pratiquement tout. Il est beaucoup utilise dans IzPack.
	\item[fork] Dans GitHub, un fork permet de creer un nouveau depot attache a son compte, dupliquant un autre depot heberge par GitHub.
	\item[W3C] World Wide Web Consortium. C'est le groupe a l'origine de la specification du XML.
\end{description}
