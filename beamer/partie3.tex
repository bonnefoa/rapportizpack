\subsection{Environnement de travail}
\begin{frame}\frametitle{Environnement de travail}
\begin{minipage}[c]{.46\linewidth}
	\begin{beamerboxesrounded}[shadow=true,center]{Gestionnaire de version}
		\centering
		\includegraphics[width=.4\linewidth]{../image/gitLogo.png}
		\includegraphics[width=.3\linewidth]{../image/githubLogo.png}
	\end{beamerboxesrounded}
\end{minipage}
\hfill
\begin{minipage}[c]{.46\linewidth}
\begin{beamerboxesrounded}[shadow=true]{Gestion de projet}
	\centering
	\includegraphics[width=.7\linewidth]{../image/lighthouseLogo.png}
\end{beamerboxesrounded}
\end{minipage}
\vfill
\hfil
\begin{minipage}[c]{.6\linewidth}
\begin{beamerboxesrounded}[shadow=true]{Environnement de développement}
	\centering
	\includegraphics[width=.4\linewidth]{../image/intellijLogo.png}
\end{beamerboxesrounded}
\end{minipage}
\end{frame}
%------------------------------------------------------------------------------
\begin{frame}\frametitle{Méthodologie agile}
\vfill
\includegraphics[width=1\linewidth]{../image/agileDev.png}
\end{frame}
%------------------------------------------------------------------------------
\subsection{Transition}
\begin{frame}
\frametitle{Patron de conception adaptateur}

% \begin{minipage}[c]{.4\linewidth}
% \begin{beamerboxesrounded}[shadow=true]{Cas initial}
% 	\centering
% 	\includegraphics[width=.8\linewidth]{../image/sansAdaptateur.png}
% \end{beamerboxesrounded}
% \end{minipage}
% \hfill
\begin{minipage}[c]{.3\linewidth}
\begin{beamerboxesrounded}[shadow=true]{Représentation}
	\centering
	\includegraphics[width=.8\linewidth]{../image/avecAdaptateur.png}
\end{beamerboxesrounded}
\end{minipage}
\hfill
\begin{minipage}[c]{.6\linewidth}
\begin{minipage}[c]{\linewidth}
\begin{beamerboxesrounded}[shadow=true]{Grandes étapes}
\begin{itemize}
	\item Extraction de l'interface
	\item Implémentation
\end{itemize}
\end{beamerboxesrounded}
\end{minipage}
~\\
~\\
\begin{minipage}[c]{\linewidth}
\begin{beamerboxesrounded}[shadow=true]{Avantages}
\begin{itemize}
	\item conservation de l'interface de NanoXML
	\item disparation de NanoXML
\end{itemize}
\end{beamerboxesrounded}
\end{minipage}
\end{minipage}
\end{frame}
%------------------------------------------------------------------------------
\subsection{Tests}
\begin{frame}\frametitle{Tests unitaires et fonctionnels}
\begin{minipage}[c]{.9\linewidth}
\begin{beamerboxesrounded}[shadow=true]{Tests unitaires}
\begin{itemize}
	\item Assurent un comportement identique
	\item Testent de nombreux cas
\end{itemize}
\end{beamerboxesrounded}
\end{minipage}
\vfill
\begin{minipage}[c]{.9\linewidth}
\begin{beamerboxesrounded}[shadow=true]{Tests fonctionnels}
\begin{itemize}
	\item IzPack
	\item Glassfish
\end{itemize}
\end{beamerboxesrounded}
\end{minipage}
\end{frame}
%------------------------------------------------------------------------------

% pas sûr de l'utilité si seulement 2 items...
\subsection{Principaux problèmes rencontrés}
\begin{frame}\frametitle{Principaux problèmes rencontrés}
\begin{beamerboxesrounded}[shadow=true]{Principaux problèmes rencontrés}
\begin{itemize}
	\item Xinclude
	\item Numéros de ligne
	\item Synchronisation des sources
\end{itemize}
\end{beamerboxesrounded}
\end{frame}
%------------------------------------------------------------------------------
