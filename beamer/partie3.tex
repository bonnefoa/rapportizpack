\subsection{Environnement de travail}
%------------------------------------------------------------------------------
\begin{frame}\frametitle{Méthode agile}
\centering
\begin{minipage}[c]{.5\linewidth}
	\begin{beamerboxesrounded}[shadow=true]{Gestion de projet}
		\includegraphics[width=\linewidth]{../image/lighthouseLogo.png}
	\end{beamerboxesrounded}
\end{minipage}
\vfill
\includegraphics[width=1\linewidth]{../image/agileDev.png}
\end{frame}
%------------------------------------------------------------------------------
\begin{frame}\frametitle{Environnement de travail}
\begin{centering}
	\begin{minipage}[c]{.6\linewidth}
		\begin{beamerboxesrounded}[shadow=true,center]{Gestionnaire de version}
			\centering
			\includegraphics[width=.4\linewidth]{../image/gitLogo.png}\hfil
		\end{beamerboxesrounded}
	\end{minipage}
	\vfill
	\begin{minipage}[c]{.6\linewidth}
		\begin{beamerboxesrounded}[shadow=true,center]{Hébergement de dépôt}
			\centering
			\includegraphics[width=.3\linewidth]{../image/githubLogo.png}
		\end{beamerboxesrounded}
	\end{minipage}
	\vfill
	\begin{minipage}[c]{.6\linewidth}
		\begin{beamerboxesrounded}[shadow=true,center]{Environnement de développement}
			\centering
			\includegraphics[width=.4\linewidth]{../image/intellijLogo.png}
		\end{beamerboxesrounded}
	\end{minipage}
	\vfill
\end{centering}
\end{frame}
%------------------------------------------------------------------------------
\subsection{Transition}
\begin{frame}
\frametitle{Patron de conception adaptateur}

\begin{minipage}[c]{.3\linewidth}
\begin{beamerboxesrounded}[shadow=true]{Représentation}
	\centering
	\includegraphics[width=.8\linewidth]{../image/avecAdaptateur.png}
\end{beamerboxesrounded}
\end{minipage}
\hfill
\begin{minipage}[c]{.6\linewidth}
\begin{minipage}[c]{\linewidth}
\begin{beamerboxesrounded}[shadow=true]{Grandes étapes}
\begin{itemize}
	\item Extraction de l'interface
	\item Implémentation
\end{itemize}
\end{beamerboxesrounded}
\end{minipage}
~\\
~\\
\begin{minipage}[c]{\linewidth}
\begin{beamerboxesrounded}[shadow=true]{Avantages}
\begin{itemize}
	\item Changements minime
	\item Conservation de l'interface de NanoXML
	\item 
\end{itemize}
\end{beamerboxesrounded}
\end{minipage}
\end{minipage}
\end{frame}
%------------------------------------------------------------------------------
\subsection{Tests}
\begin{frame}\frametitle{Tests effectués}
\begin{minipage}[c]{.9\linewidth}
Tests unitaires :
\begin{itemize}
	\item Assurent un comportement identique
	\item Testent de nombreux cas
\end{itemize}
\end{minipage}
\end{frame}
%------------------------------------------------------------------------------
\begin{frame}\frametitle{Installation de IzPack}
	\centering
	\includegraphics[width=\linewidth]{../image/installationIzpack.png}
\end{frame}
%------------------------------------------------------------------------------
\begin{frame}\frametitle{Installation de Glassfish}
	\centering
	\includegraphics[width=\linewidth]{../image/installationGlassfish.png}
\end{frame}
%------------------------------------------------------------------------------
\subsection{Principaux problèmes rencontrés}
\begin{frame}\frametitle{Principaux problèmes rencontrés}
Principaux problèmes rencontrés :
\begin{itemize}
	\item Synchronisation des sources
	\item Xinclude
	\item Numéros de ligne
\end{itemize}
\end{frame}
%------------------------------------------------------------------------------
