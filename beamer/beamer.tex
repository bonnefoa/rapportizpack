\documentclass[slidetop,11pt]{beamer}
\usepackage{pslatex}
\usepackage[T1]{fontenc} 
\usepackage[utf-8]{inputenc}
\usepackage[french]{babel}
\usepackage{listings}

\useinnertheme{default}
% izpack-logo-big.png PNG 438x150
\useoutertheme[hideothersubsections,width=2.34cm,height=0.8cm]{sidebar}

\useinnertheme{rectangles}

% \usecolortheme{seahorse}
% \usecolortheme{orchid}

\usecolortheme{seagull}

% \definecolor{craneorange}{RGB}{252,187,6}
% \definecolor{craneblue}{RGB}{4,6,76}
% 
% \setbeamercolor{block title}{fg=craneblue,bg=craneorange}
% \setbeamercolor{block title alerted}{use=alerted text,fg=craneblue,bg=alerted text.fg!75!bg}
% \setbeamercolor{block title example}{use=example text,fg=craneblue,bg=example text.fg!75!bg}
% 
% \setbeamercolor{block body}{parent=normal text,use=block title,bg=block title.bg!25!bg}
% \setbeamercolor{block body alerted}{parent=normal text,use=block title alerted,bg=block title alerted.bg!25!bg}
% \setbeamercolor{block body example}{parent=normal text,use=block title example,bg=block title example.bg!25!bg}

% \setbeamercolor{structure}{fg=black} % tout ce qui structure le doc : titres, sous titres, TOC, ...

\setbeamerfont{block title}{size={}}

\logo{\includegraphics[height=0.8cm]{../image/izpack-logo-big.png}}
% \logo{\includegraphics[height=1.1cm]{../image/izpackLogoAlpha.png}}

% \definecolor{headerTop}{HTML}{666666}
% \definecolor{headerBottom}{HTML}{3A3A3A}

\setbeamercolor{background}{bg=}
\setbeamercolor{title}{bg=}
\setbeamercolor*{titlelike}{fg=, bg=}
\setbeamercolor*{frametitle}{fg=white, bg=}
\setbeamercolor*{title}{fg=, bg=}
\setbeamercolor*{logo}{fg=, bg=}
\setbeamercolor*{sidebar}{fg=, bg=}

% moche, mais tellement pratique...
\setbeamertemplate{background canvas}{\includegraphics[width=\paperwidth,height=1.6cm]{../image/backgroundBeamer.png}}

% rappel du plan à chaque partie
\AtBeginSection[]
{
  \begin{frame}<beamer>\frametitle{Plan}
    \frametitle{}
    \tableofcontents[hideothersubsections,currentsection]
  \end{frame}
}

\title{Refactoring IzPack}
\date{\oldstylenums{\today}}

\begin{document}
%------------------------------------------------------------
\hspace{-0.15\textwidth}
\begin{frame}[plain]
\begin{center}
~
\vfill
\Large Projet de $3^{eme}$ année\\[1cm]
\hrule
~\\[0.4cm]
{ \Huge \bfseries Refactoring IzPack}\\[0.4cm]
\hrule
~\\[0.4cm]
\includegraphics[height=1.5cm]{../image/izpack-logo-big.png}
\\[0.4cm]
% \Large Filière : Génie Logiciel et Systèmes Informatiques\\
% Anthonin Bonnefoy et David Duponchel\\
% Tuteur : Julien Ponge
\vfill
\begin{minipage}{0.45\textwidth}
\begin{flushleft} %\large
\small
Anthonin \textsc{Bonnefoy}\\
David \textsc{Duponchel}\\
\end{flushleft}
\end{minipage}
\begin{minipage}{0.5\textwidth}
\begin{flushright}
% \large
\small
\emph{Encadrés par: }Julien \textsc{Ponge}\\
% {\today}
\end{flushright}
\end{minipage}\\[1cm]
\end{center}
\end{frame}
%------------------------------------------------------------
\section*{Introduction}
Avec un repo public (et bien reference en plus), j'hesite a mettre la traditionnelle introduction delirante d'avant release...
%------------------------------------------------------------
\begin{frame}\frametitle{Plan}
\tableofcontents
\end{frame}
%------------------------------------------------------------
\section{Description du problème}
%---------------------------------------------------------
\section{Description}
IzPack est un projet open-source crée en 2001 par Julien Ponge. C'est un generateur d'installeur et il presente a ce titre de nombreuses fonctionnalitees.
\subsection{Générateur d'installeur}
Une application, une fois realisee, necessite un certains nombre d'operations sur la machine pour etre operationnelle. Ces operations vont de la decompression de l'archive dans un repertoire au lancement de script de configuration en passant par la validation de la license et la creation de raccourcis. Creer un installeur pour une application est souvent laborieux et n'apporte que peu de plus-value au programme. 

L'interet d'Izpack reside dans le fait qu'il propose une solution pour creer cet installeur de maniere simple et universelle. En effet, a partir d'une application cree, Izpack est capable de generer un installateur. Cet installateur pourra etre utilise pour deployer l'application sur n'importe quelle machine. De plus, tout type de programme peut etre package, que ce soit une application C++ ou Java. Izpack se chargeant juste de la logique d'installation, il est totalement independant du contenu qu'il installe.
\subsection{Open-source}
Le projet Izpack est placee sous une license Open-source. A ce titre, le code source est librement selon les termes de la license. De plus, une communautee s'est regroupee autour du projet.
\subsubsection{La communaute Izpack}
Cette communaute open-source fait vivre et evoluer le projet. Les personnes rejoignant les projets sont des personnes aux motivations diverses. Ces personnes peuvent etre des passionnes interesse par le projet ou des salaries utilisant Izpack et apportant leurs contribution. Les formes de contributions sont variees. Elles peuvent prendre la forme d'aide aux utilisateurs, de readaction de documentation, de correction de bugs ou d'ajouts de fonctionnalites.
\subsubsection{License}
Izpack est sous license Apache 2. Cette license permet l'acces au code source et l'utilisation libre du logiciel. Il est tout a fait possible de l'utiliser pour une application commerciale voire meme, de modifier les sources pour correspondre a ses besoins. Une importante communaute s'est regroupee autour de ce projet ce qui a permit son evolution jusqu'a maintenant. Si un developpeur apporte des modifications utiles, il est encourage a en faire profiter la communaute, mais ce n'est pas une obligation.
\subsection{Fonctionnalités}
Izpack est un systeme modulaire, il possède de nombreuses fonctionnalités pour creer un installeur adapté a chacun.
\subsubsection{Multi-plateforme}
L'installeur generee par Izpack est un jar (java archive). Il suffit donc que la machine ait un machine virtuelle pour pouvoir lancer l'installation, indépendamment de la plateforme et du système d'exploitation. Il est neanmoins possibles de faire des traitements specifiques a certaines plateforme ou de convertir l'installeur pour etre specifique a une plateforme. 
\subsubsection{Personnalisation}
Izpack propose un ensemble de panels qui vont constituer l'installateur graphique. Chaque panels remplit une fonction specifique. L'aspect global et celui des panels est personnalisable par l'utilisateur via le descripteur XML.
\subsubsection{Internationalisation}
Izpack supporte la création d'installeur multilangues. Pour la localisation, tout repose sur des fichiers XML. Si une langue n'existe pas, il suffit de traduire les phrases/mots dans un fichier xml a utiliser lors de la creation de l'installeur.
\begin{figure}[H]
	\centering
	\includegraphics[width=5cm]{../image/LangChoice.png}
	\caption{Exemple de choix de langue avec IzPack}
\end{figure}
\subsubsection{Installation automatique}
A la fin de l'installation, il est possible de générer un script d'installation automatique. Ce script permet de reproduire l'installation réalisée sur d'autres machines.
\begin{figure}[H]
	\centering
	\includegraphics[width=12cm]{../image/SaveInstallXML.png}
	\caption{Fin d'installation, enregistrement du script d'installation automatisee}
\end{figure}
\subsection{Popularité du projet}
Izpack est utilisé dans de grand projets comme Jboss, Xwiki, Glassfish... A l'heure actuelle, les téléchargements mensuels s'élèvent a 25.000.
\begin{figure}[H]
	\centering
	\includegraphics[width=0.6\textwidth]{../image/telechargements.png}
	\caption{Statistiques de telechargements}
\end{figure}
 %---------------------------------------------------------
\section{Architecture}
\subsection{Architecture globale}
Globalement, Izpack possède 2 composants, le compilateur qui va gérer la création de l'installeur et l'installeur.
\subsection{Compilateur}
Le compilateur package l'ensemble des fichiers nécessaires dans un seul fichier jar. Selon la description de l'installation, il va incorporer les panels nécessaires au lancement de celle-ci. Cette partie est utilisee par le developpeur qui souhaite creer un installeur.
\subsection{Installeur}
La partie installation concerne toute la logique et la présentation du processus d'installation. Cette partie sera executer par un utilisateur souhaitant installer le logiciel.
\subsection{Processus d'installation}
Dans un premier temps, le compilateur va compiler l'application.
\begin{figure}[H]
	\centering
	\includegraphics[width=0.4\textwidth]{../image/archi_architecture.png}
	\caption{Phase de compilation}
\end{figure}
L'installateur genere comprend l'application en elle-meme et toute la partie installation necessaire et adaptee au besoin de l'application. Cette application est alors prete a etre deployee sur une autre machine.
\begin{figure}[H]
	\centering
	\includegraphics[width=0.4\textwidth]{../image/archi_installation.png}
	\caption{Phase d'installation}
\end{figure}
Le processus d'installation est gere par la partie installeur de Izpack. Cette phase va permettre de deployer l'application sur toute les machines necessaires.
\begin{figure}[H]
	\centering
	\includegraphics[width=0.3\textwidth]{../image/archi_installe.png}
	\caption{Application deployee}
\end{figure}

\subsubsection{Exemples de panels}
Chaque ecran que verra l'utilisateur est decrit par un panel.
Il existe de nombreux types de panels : des panel pour accueillir l'utilisateur et lui afficher des informations (HelloPanel et HTMLInfoPanel), d'autres pour demander à l'utilisateur des informations (UserInputPanel), etc.
Il existe aussi des panels plus spécialisés. Ainsi, CompilePanel permet de compiler du code java, et ProcessPanel permet de lancer des programmes après l'installation.

% screenshot de panels si on a du temps / de la place

 %---------------------------------------------------------
\section{Exemples d'installation}
De nombreux exemples complets existent, par exemple l'installeur de IzPack, celui de Glassfish, etc. Pour illustrer simplement l'utilisation de IzPack, utilisons plutôt le petit exemple fourni avec le code de l'application.

\subsection{Description du xml}
% un peu porc comme méthode, copier/coller une partie du xml...
% mais je ne vois pas comment présenter ce xml
% et puis ça suffira, au moins pour un premier jet

Ce xml (install.xml) décrit complètement l'installation.
Une balise \verb|info| permet de définir les informations concernant l'application : 
\begin{lstlisting}[language=xml]
<info>
	<appname>Sample Installation</appname>
	<appversion>1.4 beta 666</appversion>
	...
</info>
\end{lstlisting}
Une autre balise, \verb|guipref| permet de définir quelques propriétés de la fenêtre de l'installeur :
\begin{lstlisting}[language=xml]
<guiprefs width="640" height="480" resizable="yes"/>
\end{lstlisting}
Les langues sont définies par la balise \verb|locale| :
\begin{lstlisting}
<locale>
	<langpack iso3="eng"/>
	<langpack iso3="fra"/>
</locale>
\end{lstlisting}
Des fichiers externes nécessaires à l'installation peuvent être définis par la balise \verb|resources| :

\begin{lstlisting}[language=xml]
<resources>
	<res id="LicencePanel.licence" src="Licence.txt"/>
</resources>
\end{lstlisting}
Les panels visibles par l'utilisateur sont décrits dans la balise \verb|panels| :
\begin{lstlisting}[language=xml]
<panels>
	<panel classname="HelloPanel"/>
	...
	<panel classname="FinishPanel"/>
</panels>
\end{lstlisting}
Enfin la balise \verb|packs| contient la description des packs (les differentes parties, optionnelles ou non, de l'application) a installer.
\begin{lstlisting}[language=xml]
<packs>
	<pack name="Base" required="yes">
		<description>The base files</description>
		<file src="Readme.txt" targetdir="$INSTALL_PATH"/>
		...
	</pack>
	<pack name="Docs" required="no">
		...
	</pack>
	...
</packs>
\end{lstlisting}
Bien sûr d'autres options existent, mais celles présentées ici suffisent à créer notre installeur.
\subsection{Génération du jar}
Pour générer notre installeur, il suffit d'avoir de lancer la commande suivante : (l'executable \verb|compile| provient de IzPack)
\begin{verbatim}
$ compile install.xml
.::  IzPack - Version 4.1.0 ::.

< compiler specifications version: 1.0 >

- Copyright (c) 2001-2008 Julien Ponge
- Visit http://izpack.org/ for the latest releases
- Released under the terms of the Apache Software License version 2.0.

-> Processing  : install.xml
-> Output      : install.jar
-> Base path   : .
-> Kind        : standard
-> Compression : default
-> Compr. level: -1
-> IzPack home : .

...

Build time: Tue Mar 10 16:50:27 CET 2009
\end{verbatim}
Cette exécution va produire un fichier install.jar : notre installeur.
\subsection{Installation}
Il suffit désormais de lancer le jar pour installer notre application.
\begin{figure}[H]
	\centering
	\includegraphics[width=15cm]{../image/installSample.png}
	\caption{Exemple d'installation avec IzPack}
\end{figure}

\subsection{Installation automatique}
En lancant l'installeur avec un script d'installation automatisee en parametre, l'installation est rejouee a l'identique automatiquement.
\begin{verbatim}
$ java -jar install.jar automated.xml
[ Starting automated installation ]
[ Starting to unpack ]
[ Processing package: Base (1/3) ]
[ Processing package: Docs (2/3) ]
[ Processing package: Sources (3/3) ]
[ Unpacking finished ]
[ Writing the uninstaller data ... ]
[ Automated installation done ]
\end{verbatim}


 %---------------------------------------------------------
\section{Problèmes actuels}
IzPack possède, comme tout logiciel, des bugs potentiels ou des améliorations à effectuer.
\subsection{Code obsolete}
Izpack a debute en 2001. A cette epoque, la Jdk en etait a la version 1.3. Il y avait donc des fonctionnalitees absentes a cette epoques comme la genericite. On trouve ainsi du code comportant des tableaux d'objets ou des Vectors.
\subsection{Nanoxml}
Une amélioration possible concerne la gestion des fichiers XML, qui ont une grande importance dans IzPack.

L'abscence de processeur XML integre a la JRE 1.3 a force l'utilisation d'une librairie externe pour traiter les XML. IzPack se base donc sur nanoXML, une librairie Xml, choisie en raison de sa faible taille.
Cette librairie n'est malheureusement plus mise à jour (la dernière mise à jour date de 2003) et possède encore quelques bugs.
De plus, les versions récentes de l'environnement java (JRE) possèdent de base tout ce qu'il faut pour gérer le xml.
Se débarasser de la dépendance à nanoXML et se reposer uniquement sur la JRE permet donc non seulement de rendre la gestion des XML plus sûre et robuste, mais également de diminuer la taille des installeurs générés.

%------------------------------------------------------------
\section{Réponse au problème}
Pour diriger et faciliter l'évolution du projet, un certain nombre d'outils et de méthodes ont été mis en place au début du projet.
L'environnement avait pour but de permettre un développement agile.
On peut distinguer 3 parties : la gestion du projet, la gestion des sources et les outils de développements.
\section{Planification et gestion de projet}
La méthode agile préconise des itérations courtes avec des objectifs clairs et concis à réaliser à chaque itération.
A chaque itération, un point est fait avec le client pour définir les objectifs pour la prochaine itération, le client ici étant notre tuteur, Julien Ponge.
Nous avons utilisé un outil pour gérer ces objectifs et ces itérations, il s'agit de LightHouse.
\subsection{Lighthouse}
Lighthouse est un gestionnaire de projet en ligne.
Il se présente sous la forme d'une application web et propose toutes les fonctionnalités nécessaires à la gestions des objectifs.
On peut distinguer deux entités principales, les tickets et les sprints.
\begin{figure}[H]
	\centering
	\includegraphics[width=0.6\textwidth]{../image/lighthouse.png}
	\caption{Page d'accueil de Lighthouse}
\end{figure}
\subsection{Ticket}
Les Tickets représentent un objectif à réaliser.
Ils peuvent être crées par n'importe quel participant au projet.
Ce ticket peut ensuite être assigner à une personne et possède plusieurs états comme vu sur la figure~\ref{fig:workflow}.
\begin{figure}[H]
	\centering
	\includegraphics[width=0.2\textwidth]{../image/lighthouseWorkFlow.png}
	\caption{Workflow LightHouse}
\label{fig:workflow}
\end{figure}
Chacun de ces tickets peut être assigné à un sprint.
\subsection{Sprint}
Un sprint est un ensemble de tickets à résoudre pour une date donnée.
Il correspond à une itération dans la méthodologie agile.
Pour le projet, la durée d'un sprint était généralement de 1 ou 2 semaines.
A chaque fin de sprint, une réunion était organisée pour discuter des résultats, des reports à effectuer ou des objectifs à réaliser pour le prochains sprint.

% ----------------------------------------------------
\section{Gestionnaire des sources}
\subsection{Gestionnaire de versions}
Tout projet informatique conséquent se doit d'être sous gestionnaire de version.
En effet, un gestionnaire de version permet de suivre l'évolution du code source dans le temps.
Un dépôt conserve l'historique des changements ce qui permet de revenir à une version antérieure.
De plus, il facilite le travail collaboratif.
En effet, les développeurs travaillant de manière indépendante sur leur version du programme, la mise en commun est prise en charge par le gestionnaire.
Il permet alors de résoudre les conflits et de fusionner les fichiers modifiés.

IzPack étant un projet open-source actif, de nombreuses modifications étaient appliquées régulièrement.
Il fallait donc pouvoir travailler en parallèle des changements opérés par les contributeurs et pouvoir appliquer facilement nos modifications le jour venu.

Le dépôt de référence d'IzPack est un dépôt Subversion, les contributeurs y appliquent leurs modifications.
Cependant, pour des raisons de facilité et pour ne pas perturber le développement, il a été décidé de travailler sur un fork Git plutôt que sur le dépôt subversion.
\subsection{Git}
Git est un gestionnaire de version récent. Il a été créé en 2004 par Linus Torvald pour remplacer BitKeeper, l'ancien gestionnaire de version du noyau linux.
Il s'agit, à l'instar de BitKeeper, d'un système distribué.
Toutes les copies du dépôt sont elles-mêmes des serveurs.
Chaque personne travaille donc sur son propre dépôt et se synchronise par rapport à d'autres dépôts.
C'est de cet aspect distribué que provient une grande partie de la puissance de Git.
Un utilisateur a accès à l'historique entier une fois qu'il a cloné le dépôt.
Il peut effectuer toutes les operations voulues sur son dépôt en étant déconnecté et chaque clone est une sauvegarde intégrale du dépôt.
Ainsi, n'importe quel développeur a accès à toutes les fonctionnalités d'un gestionnaire de version même s'il n'est pas connecté au dépôt central.
La communication entre dépôts Git se fait soit par le protocole ssh, https, ftp, rsync ou par le protocole git.

Git se démarque également sur des fonctionnalités évoluées.
L'un de ses principaux points forts est la gestion des branches.
La création, suppression et fusion des branches est remarquable de simplicité et d'efficacité.
La taille d'un dépôt Git est faible comparée à celle d'autres gestionnaires comme subversion.

Julien Ponge entretient un dépôt Git synchronisé par rapport au dépôt officiel subversion.
Nous avons travaillé sur un fork de ce dépôt.
Nous pouvions à tout moment synchroniser notre dépôt par rapport au dépôt d'origine et ainsi, appliquer toutes les modifications des contributeurs à notre dépôt.
Cela permettait de préparer l'application de nos modifications de manière aisée. Tout ces dépôts sont hébergés sur GitHub.
\subsection{GitHub}
GitHub est une plateforme collaborative hébergeant des dépôts Git.
Cette plateforme permet de créer, dupliquer, supprimer ou consulter des dépôts Git.
Toutes les réalisations faites sont ouvertes et disponibles à tout le monde, c'est à dire que n'importe qui est libre de consulter, cloner ou de forker les projets présents sur GitHub.
Le droit de commit est cependant restreint aux utilisateurs choisies.
Cette plateforme a été choisie par Julien Ponge pour héberger son dépôt Git.
Nous nous sommes donc créés des comptes sur GitHub et l'avons forké.
L'utilisation d'un dépôt public basé sur celui de Julien Ponge a permis non seulement de faciliter les mises à jour, mais également de lui permettre de suivre facilement notre travail.
%- GitHub : Gestionnaire de dépôt Git
%-> Duplicat (fork) du projet
%-> Travail collaboratif et ouvert 
%-> Consultation de l'historique
%-> Visionnage des branches
\section{Outils de développement}
\subsection{IntelliJ Idea}
%-> IDE spécialisée technologies Java
%Nombreuses fonctionnalités présentes
%-> Recherche implementation 
%-> Preserve Case
%-> Refactoring puissant
IntelliJ est un environnement de développement intégré similaire à Eclipse.
Il est propriétaire et nécessite une licence pour son fonctionnement.
Cependant, JetBrain, l'éditeur de IntelliJ, a une politique d'ouverture pour les projets open-source.
Les projets sous CodeHaus et Apache bénéficient d'une licence dédiée et gratuite.

Il présente certaines divergences avec des IDE comme Eclipse et NetBeans.
Il possède une plus grande variété de fonctionnalités et l'intégration d'un grand nombre de composants.
On trouve ainsi l'intégration de frameworks récents comme GWT ou d'outils comme Git.
L'intégration de Maven est également très complète.
De plus, ses capacités de modification de code (refactoring) sont au dessus de celles des autres IDE : préservation de la casse lors d'un renommage, etc.
\subsection{Yourkit}
%Profiler Java
%-> Temps, nombres d'appels des méthodes
%-> Affichage de la charge a tout moment
Yourkit est un outil de profiling. Il permet d'avoir un rapport détaillé sur l'utilisation des ressources par le programme. On peut ainsi récupérer le nombre d'appel à une méthode, le temps passé dans une méthode, l'utilisation des ressources à un temps donné.

Le but du profiling a été de repérer les optimisations possibles sur les modifications que l'on a opéré.
%------------------------------------------------------------
\section{Réalisation de la solution}
%------------------------------------------------------------------------------
\begin{frame}\frametitle{Démonstration en vidéo}
Démonstration.
\end{frame}
%------------------------------------------------------------------------------

%------------------------------------------------------------
\section*{Conclusion}
Le projet visait à supprimer la dépendance de IzPack à NanoXML.
Cet objectif à été réalisé, la librairie NanoXML a été entièrement supprimé du projet.
Ces changements ont été placé sur la branche principale et seront disponible pour la 4.3 de IzPack, qui sortira le 24 avril.


Les tests assurent la fiabilité de la solution mais des bugs peuvent subsister.
C'est pour ça que nous continuons de suivre nos réalisations hors du cadre du projet en devenant contributeur.
De plus, nous participons désormais activement à l'évolution du projet.

Il reste encore de nombreuse possibilités d'amélioration pour IzPack.
La suppression de NanoXML en faisait partie.
Nous travaillons actuellement sur d'autres améliorations pour améliorer la qualité de IzPack.
En effet, il est préférable pour les nouveaux contributeurs de faire des petites modifications afin de b

%------------------------------------------------------------
\end{document}
